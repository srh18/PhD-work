\documentclass[12pt]{article}

\input{../Tex/header.tex}
%\geometry{twoside,bindingoffset = 2cm,left = 15mm, right = 15mm}
\onehalfspacing
\renewcommand{\bibname}{References}
% opening
\title{Flow down a cylinder of large radius and long wave crenulations}
\author[1,2]{Samuel Richard Harrison}
\author[2]{\authorcr Supervisors: Prof. D.T. Papageorgiou}

\affil[1]{ University of Reading}
\affil[2]{Imperial College London}
\date{\today}

\begin{document}

\maketitle

\begin{abstract}
	Stalactites start their life as soda straws \cite{hill1997cave}
	Soda straws have a radius that is dependent on the capillary length as the rock is deposited from a droplet \cite{curl1972minimum}. At some point external feeding occurs and this is when stalactites get their cone like shape, as external feeding has a more significant effect \cite{maltsev1999stalactites}. If crenulations on stalactites occur due to the flow down the outside of the stalactite, we should look at the flow down these initial soda straw stalactites and not the flow down stalactites in general.
\end{abstract}
\section{Introduction}
When water first starts flowing down the stalactite, it is a soda straw stalactite \cite{hill1997cave}. This is a fairly cylindrical object with a radius around the value of the capillary length \cite{curl1972minimum}. 
\section{Scalings}
The fluid thickness is however much thinner \cite{short} and so it makes sense to rescale the coordinates 
\begin{align}
\hat{r} = a_0\left(\frac{R}{\epsilon}+ a(z) r\right)\\
\hat{z} =\frac{ a_0}{\epsilon}z
\end{align}
with the hatted coordinates being the dimensional coordinates. $a_0$ is the mean fluid thickness, $\epsilon<<1$ is a small parameter denoting the ratio between the mean fluid thickness and the capillary length. The velocity field is driven by gravity so $W = \frac{g a^2}{\mu}$. We can follow steps done by Frenkel \cite{Frenkel_1992}which leads to an evolution equation 
\begin{align}
no 
\end{align}
\bibliographystyle{plain}
\bibliography{../MRES-Project/Report}
\end{document}
